\documentclass{scrreprt}

\usepackage[utf8]{inputenc}
\usepackage[ngerman]{babel}
\usepackage{amsmath}
\usepackage{amssymb}


\begin{document}

\tableofcontents

\chapter{Einleitung}
\section{Problembeschreibung}
\section{Motivation}
\section{Bisherige Ansätze}
Hier das ursprüngliche "`allumfassende"' MIP vorstellen und seine schlechte Laufzeit als Motivation für die Dekompositionsansätze nehmen.

\textit{Die Einleitung muss nur noch geschrieben werden. "`Echte"' Arbeit ist nicht mehr erforderlich.}


\chapter{1. Dekompositionsansatz}
\section{Berechnen einer Jobreihenfolge}
\textit{Dieser Abschnitt sollte in wenigen Tagen geschrieben sein.}
\subsection{Gilmore Gomory}
Hier kurz auf den Gilmore Gomory Algorithmus eingehen. Vielleicht mit einem Verweis auf Matthias Kampmeyers Arbeit?
Den Algorithmus im Detail zu erklären, ist in dieser Arbeit wohl weniger sinnvoll.

\subsection{Lineare Programmierung}
Das "`einfache"' MIP vorstellen, inklusive Laufzeiten.
Verweis auf die $\mathcal{NP}$-Schwere.


\section{Zuweisen von Ressourcen}
\subsection{Zulässigkeit der Zuweisung}
\textit{Dieser Unterabschnitt muss auch nur noch geschrieben werden.}

Zulässigkeit mit Netzflussalgorithmus nachweisen.
Darauf eingehen, dass bei disjunkten Jobgruppen keine Optimierungsmöglichkeiten bestehen.
Im Gegensatz dazu s. \ref{subsec:optres}.


\subsection{Optimierung der Ressourcenzuweisung}
\label{subsec:optres}.
\textit{Soll dieser Abschnitt noch Bestandteil der Arbeit sein? Wenn ja, dann werde ich mich da zwischendurch immer mal wieder dransetzen,
bis eine Lösung gefunden ist. Ein echter Zeitplan ist hierfür schwierig.}

Bei hierarchischen oder beliebigen Jobgruppen eine optimale Zuweisung mit möglichst wenig Rüstkosten finden.
Dazu (falls möglich) einen polynomiellen Algorithmus angeben bzw. einen Beseis zur $\mathcal{NP}$-Vollständigkeit.

\section{Unzulässige Reihenfolgen}
\subsection{Nachbarschaftssuche}
\textit{Anfang August hiermit anfangen.}

Nachbarschaftenssuchen formulieren, die eine unzulässige Reihenfolgen reparieren, so dass sie dann zulässig ist.
Dabei soll die Reihenfolge möglichst wenig von ihrer Optimalität einbüßen.

\subsection{Andere Ansätze}
Platzhalter, falls noch andere Ideen aufkommen.


\chapter{2. Dekompositionsansatz}
\section{Ressourcenzuweisung}
\textit{Dieser Abschnitt muss auch nur noch ausformuliert werden.}

Generell auf bekannte Laufzeitschranken und Verfahren für die Ressourcenzuweisung kurz eingehen.
Davon speziell erklären, wie mit Binpacking eine minimale Anzahl an Rüstkosten erreicht werden kann für $s_{fg}=s$.


\section{Zulässigkeit der Zuweisung}
\textit{Bis Mitte Juli sollte die Lösung hier erarbeitet sein.}

Folgende Fragen diskutieren und idealer Weise beweisen:
Gibt es unzulässige Lösungen beim Binpacking? Wenn ja, wie lassen sie sich reparieren?


\section{Anordnung der Jobgruppen}
\subsection{MIP mit fixierten Bins}
Vorstellung des MIPs mit fixierten Bins mit $s_{fg}=s$. Vorstellung einiger Laufzeiten. 
\textit{MIP steht. Evtl müssen noch einige Laufzeiten gemessen werden.}

Außerdem die erweiterte Formulierung auf allgemeine $s_{fg}$ und Diskussion.
\textit{MIP sollte bis Ende Juli formuliert und Laufzeiten gemessen sein.}

\subsection{MIP mit freien Bins}
\label{subsec:mipfreibins}
\textit{Muss nur noch aufgeschrieben werden. Evtl. noch ein paar Laufzeiten messen.}

Vorstellung des MIPs mit freien Bins. 
Die Laufzeit ist sehr viel schlechter, die primale Lösung ist aber meist schon nach kurzer Zeit besser als beim MIP mit fixierten Bins.

\subsection{Mehrere fixierte Reihenfolgen}
\textit{Jetzt damit anfangen. Wahrscheinlich bis Mitte August.}

Basierend auf den Erkenntnissen aus \ref{subsec:mipfreibins} das MIP mit fixierten Bins für mehrere Binreihenfolgen laufen lassen.
Um nicht sämtliche Binreihenfolgen durchprobieren zu müssen,
ggf. mit bipartitem gewichteten Matching ein lokales Optimum zwischen zwei Bins suchen und daraus eine "`gute"' Binreihenfolge erstellen.


\chapter{Messergebnisse und Vergleiche}
\textit{Anfang September anfangen, soweit Messergebnisse vorliegen.}

Obwohl in den vorherigen Kapiteln schon einige Laufzeiten vorgestellt werden, hier nochmal eine Zusammenfassung und insbesondere ein Vergleich
zwischen den beiden Dekompositionsansätzen. Sowohl echte Instanzen als auch generierte. 
Dabei ggf. auf Methoden zur Instanzengenerierung eingehen und diese bewerten?

\end{document}
