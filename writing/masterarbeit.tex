\documentclass{scrreprt}

\usepackage[utf8]{inputenc}
\usepackage[ngerman]{babel}
\usepackage{amsmath}
\usepackage{amssymb}
\usepackage{mathtools}

\parskip 12pt

\begin{document}

\tableofcontents

\chapter{Einleitung}
\section{Problembeschreibung}
Bei synchronen Flow-Shop-Problemen handelt es sich um Produktions\-planungs\-probleme,
bei denen die zu produzierenden Güter (Jobs) auf einer zyklisch angeordneten Produktionsanlage
produziert werden. Die Produktionsanlage be\-steht aus $m$ Stationen, die sich mit der Anlage drehen.
Außen, um die Anlage herum, befinden sich $m$ fortlaufend nummerierte fixierte Maschinen $M_1,\ldots,M_m$, die die einzelnen Produktionsschritte durchführen.
Dabei handelt es sich bei Maschine $M_1$ um das Einlegen des Jobs in die Anlage und bei Maschine $M_m$ um die Entnahme des fertigen Produkts.
Durch Rotation der Anlage werden die Stationen mit den auf ihnen befindlichen Jobs zur jeweils nächsten Maschine transportiert.
Die Reihenfolge, in der alle Jobs die Maschinen durchlaufen müssen ist also fest vorgegeben.
Eine Rotation darf immer nur dann stattfinden, wenn alle Maschinen ihren Produktionsschritt an ihrem aktuellen Job
durchgeführt haben. Auf diese Weise können die Jobs, im Gegensatz zum klassischen (asynchronen) Flow-Shop, 
immer nur \textit{synchron} zur nachfolgenden Maschine gelangen,
Die Zeit, die zwischen zwei Rotationen vergeht, wird als \textit{Zykluszeit} bezeichnet.

Die zu produzierenden Jobs sind gegeben durch die Menge $J=\{j_1,\ldots,j_n\}$ 
und die Prozesszeiten von Job $j$ auf Maschine $i$ sind durch $p_{ij}$ gegeben.
Ziel ist es, eine Reihenfolge $\pi\in J^n$ der Jobs zu erstellen, die die gesamte Produktionsdauer minimiert.
Die Zykluszeiten $c_t$ mit $1\leq t\leq n+m-1$ berechnen sich wie folgt:
\[ c_t = \max_{i=\max\{1,t-n+1\}}^{\min\{t,m\}} p_{i\pi_{t-i+1}} \]
Die Zielfunktion ist also $C_{\max} = \sum_{t=1}^{n+m-1} c_t$. 
Andere Zielfunktionen werden in dieser Masterarbeit nicht betrachtet.

Eine Teilmenge $D \subseteq \{M_1,\ldots,M_m\}$ der Maschinen heißt \textit{dominierend}, wenn 
\[ p_{dj} \geq p_{ej} \quad \forall j\in J, d\in D, e\not\in D \] 
ist. Die Prozesszeiten aller Jobs auf dominierenden Maschinen sind also immer mindestens so groß wie die Prozesszeiten auf den restlichen Maschinen.
Treten dominierende Maschinen auf, müssen für die Berechnung der Zykluszeiten die Prozesszeiten auf den übrigen Maschinen also nicht betrachtet werden.

Zusätzlich benötigen die Jobs Ressourcen aus einer Menge $R$, um in die Stationen eingelegt werden zu können. Diese Ressourcen können erst nach
Fertigstellung eines Jobs, also nachdem er an Maschine $M_m$ aus der Anlage genommen wurde, wiederverwendet werden.
Sie sind allerdings nur in begrenzter Zahl vorhanden und im Allgemeinen ist nicht jede Ressource für jeden Job geeignet.
Für $j\in J$ sei $\rho_j\subseteq R$ die Menge der Ressourcen, die für $j$ geeignet ist.
Umgekehrt sei für $r\in R$ mit $\iota_r\subseteq J$ die Menge der Jobs bezeichnet, für die $r$ geeignet ist.
An Maschine $M_1$ kann es daher notwendig sein, vor dem Einlegen des nächsten Jobs die Ressource zu wechseln, 
wenn auf der entsprechenden Station zuvor Job $j\in J$ mit Ressource $r\in\rho_j$ fertiggestellt wurde 
und nun Job $j'\in J$ eingelegt werden soll, wobei $r\not\in\rho_{j'}$ ist.

Für die Ressourcen können folgende Situationen auftreten:
\begin{itemize}
    \item Alle Ressourcen sind für alle Jobs geeignet, also $\rho_j=R$ für alle $j\in J$.
    \item Die Jobs lassen sich in disjunkte Gruppen unterteilen, so dass für alle Jobs aus einer Gruppe dieselbe Ressourcenmenge geeignet ist.
        Wenn also $\rho_i \cap \rho_j \neq \emptyset$, dann folgt $\rho_i=\rho_j$.
    \item Die Ressourcenmengen bilden Hierarchien. 
        D.h., wenn $\iota_q \cap \iota_r \neq \emptyset$, dann folgt $\iota_q \subseteq \iota_r$ oder $\iota_r \subseteq \iota_q$.
    \item Die $\rho_j$ sind beliebige Teilmengen von $R$.
\end{itemize}%
Neben dem Wechsel von Ressourcen, der zusätzliche Zeit in Anspruch nimmt, können auch andere Formen von \textit{Rüstkosten}
auftreten. Z.B. kann es sein, dass an einer Station zunächst einige Umstellungen vorgenommen werden müssen, bevor der
neue Job eingelegt werden kann. Die Jobs können in Familien $\mathcal{F}$ eingeteilt werden, so dass beim Übergang
zwischen zwei Jobs aus den Familien $f$ und $g$ die Rüstkosten $s_{fg}$ auftreten.
Diese Rüstkosten können 
\begin{itemize}
    \item sowohl vom Vorgänger als auch vom Nachfolger abhängig sein ($s_{fg}$), 
    \item nur vom Nachfolger abhängig sein ($s_{fg} = s_{g}$) oder
    \item konstant sein ($s_{fg} = s > 0$).
\end{itemize}%
Dabei wird jeweils für $s_{ff} = 0$ angenommen, dass also keine Rüstkosten innerhalb einer Familie auftreten.

Insgesamt gilt es also, neben der Reihenfolge $\pi$ auch ein Mapping $f:J\rightarrow R$ zu finden, 
das jedem Job $j\in J$ eine Ressource $r\in\rho_j$ zuweist und folgenden Ansprüchen genügt:
\begin{itemize}
    \item $f$ muss zulässig sein in dem Sinne, dass beim Einlegen jedes Jobs $j\in J$ eine Ressource $r\in\rho_j$ verfügbar ist
        (d.h., dass $r$ sich nicht gerade an anderer Stelle in der Anlage befindet).
    \item $f$ und $\pi$ zusammen sollen optimal sein in dem Sinne, dass die Summe aus den durch $\pi$ und $f$ definierten Zykluszeiten und 
        Rüstkosten minimal ist.
\end{itemize}

\section{Motivation}
In \cite{...} wurde gezeigt, dass dieses Problem schon $\mathcal{NP}$-schwer ist, wenn alle Ressourcen für alle Jobs geeignet sind
und keine Rüstkosten auftreten.
Versuche, dieses Problem -- oder auch einige Spezialfälle davon -- mit linearer Programmierung zu lösen, waren nur für sehr kleine Instanzen mit
$n<30$ erfolgreich, was weit hinter praktischen Anforderungen zurückliegt. Aufgrund der Komplexität des Problems sollen in dieser Arbeit
zwei Dekompositionsansätze verfolgt werden:
\begin{enumerate}
    \item Zunächst wird eine Reihenfolge $\pi$ aufgestellt, ohne Ressourcen und Rüstkosten zu betrachten, 
        und anschließend wird das Mapping $f$ basierend auf $\pi$ erstellt, ohne $\pi$ noch zu verändern.
    \item Es wird zuerst das Mapping $f$ erstellt, so dass die Ressourcen zulässig zugewiesen sind und die Rüstkosten minimal sind.
        Anschließend werden, ohne $f$ zu verändern, die Jobs so angeordnet, dass die Zykluszeiten möglichst minimal sind, und so $\pi$ erstellt.
\end{enumerate}%
Beide Ansätze liefern natürlich im Allgemeinen keine optimalen Lösungen. 
Außerdem sind selbst die aus den Ansätzen resultierenden Teilprobleme teilweise noch $\mathcal{NP}$-schwer.
Beispielsweise ist bei Ansatz (1) das Berechnen einer optimalen Reihenfolge $\pi$ (soweit bekannt) nur in dem Spezialfall,
dass es genau zwei benachbarte dominierende Maschinen gibt,
mit dem Algorithmus von Gilmore und Gomory \cite{...} in Polynomialzeit lösbar.
Bei der anschließenden Zuweisung von Ressourcen ist noch unbekannt, ob ein polynomieller Algorithmus existiert.
%Dies herauszufinden ist eines der Ziele dieser Arbeit.

Da die Rüstkosten $s_{fg}$ in dem gegebenen Praxisfall deutlich größer sind als die Prozesszeiten $p_{ij}$ der Jobs auf den Maschinen,
ist davon auszugehen, dass Ansatz (2) für diesen Praxisfall die besseren Lösungen erzielen wird.
Nichtsdestotrotz ist auch der erste Ansatz interessant, da er in anderen Praxisbeispielen von Bedeutung sein kann.


\section{Bisherige Ansätze}
\label{sec:bisherigeAnsaetze}
Als weitere Motivation für die Dekompositionsansätze dient folgendes Mixed Integer Linear Program (MIP):
\begin{align}
    \text{min} \quad \sum_{t=1}^{n+m-1} &c_t + \sum_{j=1}^n \sum_{h=1}^n s_{jh} y_{jh} \label{allmip:obj}\\
    \text{s.t.}\quad \sum_{k=1}^n x_{jk} &= 1 & j\in N \label{allmip:1}\\
                     \sum_{j=1}^n x_{jk} &= 1 & k\in N \label{allmip:2}\\
    c_t &\geq \sum_{j=1}^n p_{t-k+1,j} \cdot x_{jk} & t\in T, k=\max\{1,t-m+1\},\ldots,\min\{n,t\} \label{allmip:3}\\
    y_{jh} + 1 &\geq x_{j,k-m} + x_{hk} & j,h\in N, k=m+1,\ldots,n \label{allmip:4}\\
    c_t &\geq 0 & t\in T \\
    x_{jk} &\in \{0,1\} & j,k\in N \\
    y_{jh} &\in \{0,1\} & j,h\in N
\end{align}
$N$ ist die Indexmenge der Jobs, also $N=\{1,\ldots,n\}$ und $T=\{1,\ldots,n+m-1\}$ ist die Indexmenge der Zykluszeiten.
Die Binärvariablen $x_{jk}$ geben an, an welcher Position $k$
sich Job $j$ befindet. Es gilt $x_{jk}=1$ genau dann, wenn $j$ an
Position $k$ ist. Die Nebenbedingungen \ref{allmip:1} und \ref{allmip:2}
stellen sicher, dass sich jeder Job an genau einer Position befindet und 
dass sich an jeder Position genau ein Job befindet.
In Nebenbedingung \ref{allmip:3} werden die Zykluszeiten bestimmt.
Den Binärvariablen $y_{jh}$ wird in Nebenbedingung \ref{allmip:4} folgende
Bedeutung gegeben: $y_{jh}=1$, wenn Job $j$ genau $m$ Positionen
vor Job $h$ in $\pi$ liegt.
In der Zielfunktion \ref{allmip:obj} wird dann die Summe aus allen
Zykluszeiten gebildet und die Summe aller Rüstkosten $s_{jh}$, die beim
Übergang von Job $j$ zu $h$ auftreten.

Obwohl in diesem MIP nicht die Verfügbarkeit von Ressourcen beachtet werden,
benötigt es schon bei $n=20$ mehrere Stunden zum Finden der optimalen Lösung.
Es ist also für Instanzen mit mehreren Tausend Jobs nicht geeignet.
Aufgrund dieser Tatsache ist eine heuristische Herangehensweise an 
synchrone Flow-Shop-Probleme mit Ressourcen und Rüstkosten eine notwendige Alternative.


\chapter{Der erste Dekompositionsansatz}
Beim ersten Dekompositionsansatz wird zunächst eine (möglichst optimale) Jobreihenfolge $\pi$ bestimmt.
Dabei werden Ressourcen und Rüstkosten außer Acht gelassen. 
Es wird also zunächst ausschließlich $C_{\max}$ optimiert. Anschließend wird ein Mapping $f$ aufgestellt,
so dass Ressourcen nur dann eingeplant werden, wenn sie auch zur Verfügung stehen, und darüber hinaus
möglichst selten ausgetauscht werden müssen, so dass geringe Rüstkosten auftreten.

In Abschnitt \ref{sec:BerechnenEinerJobreihenfolge} werden einige exakte und heuristische Verfahren für die
Berechnung von $\pi$ vorgestellt, was im Allgemeinen $\mathcal{NP}$-schwer ist.
Anschließend werden in Abschnitt \ref{sec:ZuweisungVonRessourcen} Verfahren vorgestellt, die ein möglichst gutes Mapping $f$ erzeugen.
Dabei wird speziell darauf eingegangen, ob mit der gegebenen Reihenfolge $\pi$ und den gegebenen Ressourcen
überhaupt eine zulässige Lösung möglich ist, und darauf, wie dann ggf. die Zulässigkeit durch nachträgliche Änderungen an $\pi$
erzeugt werden kann.

\section{Berechnen einer Jobreihenfolge}
\label{sec:BerechnenEinerJobreihenfolge}
\subsection{Gilmore Gomory}
Der Algorithmus von Gilmore und Gomory \cite{...} löst in $\mathcal{O}(n\log n)$ einen speziellen Fall des gerichteten Travelling-Salesman-Problems (TSP),
bei dem alle Knoten zwei Parameter $x,y$ haben und die Kantenkosten $c_{ij}$ zwischen je zwei Knoten $i$ und $j$ nur vom $x$-Wert von $i$
und vom $y$-Wert von $j$ abhängig sind.

Diese Situation liegt beim synchronen Flow-Shop vor, wenn es nur zwei benachbarte dominierende Maschinen gibt.
O.B.d.A seien dies $M_1$ und $M_2$.
Jobs können durch Knoten repräsentiert werden und die beiden Prozesszeiten auf den dominierenden Maschinen liefern die Parameter $x$ und $y$. 
Der Abstand zwischen zwei Knoten entspricht dann der Zykluszeit, die die entsprechenden Jobs verursachen, wenn sie nebeneinander liegen.
Die Berechnung der Zykluszeiten vereinfacht sich hier zu
$c_t = \max \{p_{2\pi_{t-1}},p_{1\pi_t}\}$ für $2\leq t\leq n$.
Sie sind also für je zwei Jobs $\pi_{t-1},\pi_{t}$ nur noch von der Prozesszeit des vorderen Jobs auf der zweiten Maschine ($p_{2\pi_{t-1}}$) 
und der Prozesszeit des hinteren Jobs auf der ersten Maschine ($p_{1\pi_t}$) abhängig.

Auf die Funktionsweise des Algorithmus soll in dieser Arbeit nicht näher eingegangen werden.


\subsection{Lineare Programmierung}
\label{subsec:LineareProgrammierung}
Zum Finden einer optimalen Lösung dieses Teilproblems wurde folgendes (MIP) aufgestellt.
Es ist identisch mit dem MIP in Abschnitt \ref{sec:bisherigeAnsaetze} bis auf die Nichtberücksichtigung der Rüstkosten.
\begin{align}
    \text{min} \quad \sum_{t=1}^{n+m-1} &c_t \\
    \text{s.t.}\quad \sum_{k=1}^n x_{jk} &= 1 & j\in N \\
                     \sum_{j=1}^n x_{jk} &= 1 & k\in N \\
    c_t &\geq \sum_{j=1}^n p_{t-k+1,j} \cdot x_{jk} & t\in T k=\max\{1,t-m+1\},\ldots,\min\{n,t\} \\
    c_t &\geq 0 & t\in T \\
    x_{jk} &\in \{0,1\} & j,k\in N
\end{align}
Dieses MIP liefert für Instanzen mit $n\leq 30$ eine optimale Lösung in unter einer Stunde.
Diese Laufzeiten sind zwar schon deutlich besser als die Variante mit Rüstkosten in Abschnitt \ref{sec:bisherigeAnsaetze},
bei größeren Instanzen ist dieser Zeitaufwand allerdings immer noch nicht praktikabel. 


\subsection{Heuristische Verfahren}
\label{subsec:HeuristischeVerfahren}
Aufgrund der $\mathcal{NP}$-Schwere der Optimierung von $C_{\max}$ und der schlechten Laufzeit des MIPs aus Abschnitt \ref{subsec:LineareProgrammierung}
bei Instanzen realer Größe werden hier einige heuristische Ansätze zur Berechnung von $C_{\max}$ vorgestellt.

\subsubsection{Non-Full-Schedule-Heuristik}
Diese Heuristik ist eine konstruktive Greedy-Heuristik, die Schritt für Schritt einen Job an $\pi$ anhängt, beginnend mit einer leeren Reihenfolge.
In jeder Iteration werden alle noch verbleibenden Jobs bewertet und der Job mit der besten Bewertung wird an $\pi$ angehängt.
Die Heuristik benötigt also genau $n$ Iterationen.
Die Bewertungsfunktion betrachtet die letzten $m-1$ Zykluszeiten der noch nicht fertigen Reihenfolge, 
wobei die Zykluszeiten am Anfang bei einer noch leeren Reihenfolge als $0$ angenommen werden.
Für jeden Job $j$, der noch nicht in $\pi$ ist, werden diese $m-1$ Zykluszeiten mit den ersten $m-1$ Prozesszeiten von $j$ verglichen.
Die Idee ist, dass diese möglichst übereinstimmen sollten. Ist eine Prozesszeit sehr viel größer als die aktuelle Zykluszeit,
zu der sie hinzugefügt werden würde, würde die Zykluszeit entsprechend um einen Wert $c_+$ ansteigen.
Ist umgekehrt die Zykluszeit sehr viel größer als die zugehörige Prozesszeit von $j$, dann würde diese kurze Prozesszeit verschenkt werden.
Die Differenz aus Zykluszeit und Prozesszeit wird mit $c_-$ bezeichnet.
Die Berwertungsfunktion berechnet für jeden Job die Summe aus den $m-1$ $c_+$-Werten.
Diese Summe ist die Bewertung für einen Job $j$. Der Job mit der kleinsten Bewertung wird an $\pi$ angehängt.
Falls mehrere Jobs eine optimale Bewertung haben, wird für diese Jobs als zweites Kriterium die Summe der $c_-$-Werte betrachtet.
Der Job, bei dem diese Summe am kleinsten ist, verschenkt am wenigsten Zeit und wird ausgewählt.
Da in jeder Iteration alle Verbleibenden Jobs betrachtet werden und für jeden dieser Jobs $m-1$ Zeiten Verglichen werden,
liegt die asymptotische Laufzeit dieser Heuristik in $\mathcal{O}(n^2m)$.

\subsubsection{Double Ended Non-Full-Schedule-Heuristik}
Diese Heuristik basiert auf der Non-Full-Schedule-Heuristik.
Der Unterschied besteht darin, dass $\pi$ nicht nur von vorne, sondern gleichzeitig auch von hinten zur Mitte hin aufgebaut wird.
Jeder noch nicht in $\pi$ enthaltende Job wird pro Iteration mit beiden Enden der bisherigen Reihenfolge verglichen.
Die Bewertungsfunktion für das hintere Ende arbeitet analog.
Es wird der beste Job für das vordere Ende und der beste für das hintere Ende gesucht und der mit der besseren Bewertung wird vorne bzw. hinten eingefügt.
Die asymptotische Laufzeit liegt hier ebenfalls in $\mathcal{O}(n^2m)$.

\subsubsection{Gilmore-Gomory-Heuristik}
Diese Heuristik wendet den Algorithmus von Gilmore und Gomory auf beliebige Instanzen an.
Dazu wird eine Instanz $I$ zunächst auf dominierende Maschinen untersucht.
Nun können zwei Fälle eintreten:
\begin{enumerate}
    \item Unter den dominierenden Maschinen $D$ gibt es zwei, die benachbart sind, also $\exists i\in\{1,\ldots,m\}, M,M'\in D$ mit $M=M_i$ und $M'=M_{i+1}$.
    \item Es gibt keine benachbarten dominierenden Maschinen.
\end{enumerate}
Wenn Fall (2) eintritt, werden alle $m$ Maschinen als dominierend angesehen, da dies für den Algorithmus keine Einschränkung darstellt.
Nun werden zwei benachbarte dominierende Maschinen gewählt.
O.B.d.A seien dies die Maschinen $M_1$ und $M_2$. Seien 
\begin{align} 
    d_{\min} &\coloneqq \min_{\substack{i\in \{M_1,M_2\} \\ j\in J}} p_{ij} \\
    e_{\max} &\coloneqq \max_{\substack{i\in D\setminus\{M_1,M_2\} \\ j\in J}} p_{ij} \\
    K &\coloneqq \frac{e_{\max}}{d_{\min}} \text{.}
\end{align}
Nun wird aus der gegebenen Instanz $I$ eine neue Instanz $I'$ erzeugt, die sich nur dadurch von $I$ unterscheidet, dass die Prozesszeiten
auf allen dominierenden Maschinen außer auf $M_1$ und $M_2$ mit $\frac{1}{K}$ skaliert werden, also
\begin{align}
    p'_{ij} \coloneqq \begin{cases} \frac{p_{ij}}{K} &\text{für } i\in D\setminus\{M_1,M_2\} \\ p_{ij} &\text{sonst.} \end{cases}
\end{align}
Auf diese Weise sind $M_1$ und $M_2$ in $I'$ die einzigen dominierenden Maschinen und folglich kann $I'$ mit dem Algorithmus von Gilmore und Gomory optimal gelöst werden.
Die resultierende Reihenfolge $\pi$ wird als heuristische Lösung für die ursprüngliche Instanz $I$ verwendet.

Da die Prozesszeiten der in $I'$ vernachlässigten dominierenden Maschinen sich um den Faktor $K$ von denen in $I$ unterscheiden, 
können sich die aus $\pi$ ergebenen Zykluszeiten von $I$ und $I'$ auch maximal um den Faktor $K$ unterscheiden:
\begin{align}
    c_t \leq K\cdot c'_t \quad \forall 1\leq t\leq n+m-1
\end{align}
Da die optimale Lösung von $I'$ eine untere Schranke für die optimale Lösung von $I$ ist, also $C'_{\max} \leq C_{\max}$, folgt für die Lösung $L_{GG}$ der
Gilmore-Gomory-Heuristik:
\begin{align}
    L_{GG} \leq K\cdot C'_{\max} \leq K\cdot C_{\max}
\end{align}
Die Gilmore-Gomory-Heuristik hat also eine relative Gütegarantie von $K$.
Zusätzlich kann bei der Wahl der zwei benachbarten dominierenden Maschinen der Parameter $K$ optimiert werden, indem nicht beliebige Maschinen gewählt werden,
sondern solche mit optimalen $d_{\min}$- bzw. $e_{\max}$-Werten.

Das finden der geeigneten dominierenden Maschinen kann in $\mathcal{O}(nm)$ durchgeführt werden. 
Das Anpassen der Prozesszeiten der übrigen dominierenden Maschinen ist nur in der Theorie von Interesse.
Der Gilmore-Gomory-Algorithmus kann diese einfach ignorieren.
Die Gesamtlaufzeit beträgt daher $\mathcal{O}(nm + n\log n)$.

Im Rahmen dieser Heuristik soll der Begriff der \textit{Semidominanz} eingeführt werden.
Die Semidominanz ist ein Maß dafür, wie weit eine Teilmenge von Maschinen davon entfernt ist, dominierend zu sein.
Eine Semidominanz von $0$ bedeutet, die Maschinen sind dominierend.
Für eine Teilmenge von Maschinen $D\subseteq\{M_1,\ldots,M_m\}$ sei 
\[d_{\min} = \min_{\substack{i:M_i\in D \\ j\in\{1,\ldots,n\}}} p_{ij}\] 
die kleinste ihrer Prozesszeiten, analog zur obigen Definition von $d_{\min}$.
Die Semidominanz ist dann definiert als
\[\mathcal{D}_{D} \coloneqq \sum_{i: M_i\not\in D} \sum_{j=1}^n \max\{0,p_{ij}-d_{\min}\} \text{.}\]
Unabhängig von der relativen Gütegarantie $K$ liefert diese Heuristik auch eine absolute Gütegarantie von $\mathcal{D}_{\{M_1,M_2\}}$, 
wenn $M_1$ und $M_2$ als benachbarte Maschinen ausgewählt werden.

\subsubsection{Nachbarschaftssuche}
Mit einer simplen Nachbarschaftssuche können bereits existierende (nicht optimale) Reihenfolgen verbessert werden.
Mögliche Nachbarschaftsoperatoren sind
\begin{itemize}
    \item $\mathit{xch}_{ij}$: vertauscht die Jobs an den Positionen $i$ und $j$ miteinander,
    \item $\mathit{shift}_{ij}$: "`shiftet"' den Job an Position $i$ nach Position $j$. 
        Alle Jobs zwischen $i$ und $j$ rücken um eins nach links (falls $i<j$) bzw. nach rechts (falls $i>j$).
\end{itemize}
Auf Grundlage dieser Nachbarschfaften lassen sich z.B. Iterative Improvement, eine Tabu-Suche oder Simulated Annealing implementieren.


\subsection{Vergleich der Heuristiken}



\section{Zuweisung von Ressourcen}
\label{sec:ZuweisungVonRessourcen}
In diesem Abschnitt geht es darum, den Jobs in der Reihenfolge $\pi$, die in Abschnitt \ref{sec:BerechnenEinerJobreihenfolge} aufgestellt wurde,
Ressourcen zuzuweisen. An diese Zuweisung werden zwei Anforderungen gestellt:
\begin{enumerate}
    \item Die Zuweisung muss zulässig sein. Das heißt, ist eine Ressource einem Job zugewiesen, darf sie den nachfolgenden $m-1$ Jobs nicht mehr zugewiesen werden.
        Das Problem der Zulässigkeit einer Zuweisung wird im ersten Unterabschnitt \ref{subsec:ZulaessigkeitDerZuweisung} betrachtet.
    \item Die Zuweisung soll möglichst optimal sein. Wenn zwei Jobs, die im Abstand von $m$ in $\pi$ liegen, die selbe Ressource benutzen können,
        werden Rüstkosten eingespart, da die Ressource nicht ausgetauscht werden muss.
        Dieses Problem wird in Unterabschnitt \ref{subsec:OptimierungDerRessourcenzuweisung} diskutiert.
\end{enumerate}
Es ist natürlich möglich, dass bei gegebenem $\pi$ keine zulässige Zuweisung von Ressourcen möglich ist.
In diesem Fall muss $\pi$ nachträglich geändert werden, worauf in Unterabschnitt \ref{UnzulaessigeReihenfolgen} eingegangen wird.

\subsection{Zulässigkeit der Zuweisung}
\label{subsec:ZulaessigkeitDerZuweisung}
Abhängig davon, welche Ressourcen für welche Jobs geeignet sind, kann das Zulässigkeitsproblem unterschiedlich schwer zu lösen sein.
Folgende Situationen werden betrachtet:
\begin{itemize}
    \item Alle Ressourcen sind für alle Jobs geeignet (trivial, es müssen $m$ Ressourcen vorhanden sein).
    \item Die Ressourcenmengen sind disjunkt ($\rho_i\cap\rho_j \neq \emptyset \enspace \Rightarrow \enspace \rho_i = \rho_j$).
    \item Die Ressourcen sind für hierarchische Jobgruppen geeignet 
        ($\iota_q\cap\iota_r \neq \emptyset \enspace \Rightarrow \enspace \iota_q\subseteq\iota_r$ oder $\iota_r\subseteq\iota_q$).
    \item Die $\rho_i$ sind beliebige Teilmengen von $R$.
\end{itemize}

\subsubsection{Zulässigkeit bei disjunkten Ressourcenmengen}
In dieser Situation können die Jobs in Gruppen unterteilt werden, so dass zu jeder Jobgruppe eine für sie exklusive Menge an zulässigen Ressourcen zur Verfügung steht.
Es reicht also aus, statt der Ressourcenmengen $\rho_i$ nur deren Größen $|\rho_i|$ zu betrachten.
Mit einem Greedy-Algorithmus kann eine gegebene Reihenfolge $\pi$ dann auf Zulässigkeit überprüft werden:
Der Algorithmus durchläuft $\pi$ von vorne nach hinten.
Jede Jobgruppe erhält einen Counter, der mitzählt, wie viele Ressourcen momentan für sie zur Verfügung stehen.
Diese Counter werden zu Beginn mit $|\rho_i|$ initialisiert.
An jeder Position in $\pi$ wird der der Counter der zugehörigen Jobgruppe dekrementiert.
Nach $m$ Schritten wird dieser Counter wieder inkrementiert.
Sollte ein Counter einmal negativ werden, gibt es keine zulässige Ressourcenzuteilung.

\subsubsection{Zulässigkeit bei hierarchischen Jobgruppen}
%Diese Situation kann mit einem ähnlichen Greedy-Algorithmus gelöst werden, wie bei disjunkten Ressourcenmengen.
%Die Ressourcen, die für eine Jobgruppe geeignet sind, können hier allerdings nicht als identisch angesehen werden,
%so dass es genügen würde, wieder nur die Anzahl der Ressourcen zu betrachten.
%Die Ressourcen unterscheiden sich dahingehend voneinander, dass sie für unterschiedlich viele andere Jobs ebenfalls geeignet sind.
%Der Greedy-Algorithmus muss so abgeändert werden, dass er jedem Job in $\pi$ immer die "`speziellste"' aktuell für diesen Job verfügbare Ressource zuweist.
%Soll einem Job $j$ eine Ressource zugeteilt werden, muss unter allen geeigneten Ressourcen $r\in\rho_j$, 
%wobei $r$ nicht schon einem anderen Job zugewiesen sein darf, diejenige mit minimalem $|\iota_r|$ ausgesucht werden.
%
%Dieser abgewandelte Greedy-Algorithmus ist korrekt, denn immer, wenn die Situation auftritt, dass für einen Job keine Ressource verfügbar ist,
%liegt das daran, dass alle potentiell geeigneten Ressourcen


\subsection{Optimierung der Ressourcenzuweisung}
\label{subsec:OptimierungDerRessourcenzuweisung}
\textit{Soll dieser Abschnitt noch Bestandteil der Arbeit sein? Wenn ja, dann werde ich mich da zwischendurch immer mal wieder dransetzen,
bis eine Lösung gefunden ist. Ein echter Zeitplan ist hierfür schwierig.}

Bei hierarchischen oder beliebigen Jobgruppen eine optimale Zuweisung mit möglichst wenig Rüstkosten finden.
Dazu (falls möglich) einen polynomiellen Algorithmus angeben bzw. einen Beweis zur $\mathcal{NP}$-Vollständigkeit.

\section{Unzulässige Reihenfolgen}
\label{UnzulaessigeReihenfolgen}
Für den Fall, dass für eine gegebene Reihenfolge $\pi$ keine Ressourcenzuweisung möglich ist, muss $\pi$ nachträglich verändert werden.
\subsection{Nachbarschaftssuche}
\textit{Anfang August hiermit anfangen.}

Nachbarschaftenssuchen formulieren, die eine unzulässige Reihenfolgen reparieren, so dass sie dann zulässig ist.
Dabei soll die Reihenfolge möglichst wenig von ihrer Optimalität einbüßen.

\subsection{Andere Ansätze}
Platzhalter, falls noch andere Ideen aufkommen.






\chapter{2. Dekompositionsansatz}
\section{Ressourcenzuweisung}
\textit{Dieser Abschnitt muss auch nur noch ausformuliert werden.}

Generell auf bekannte Laufzeitschranken und Verfahren für die Ressourcenzuweisung kurz eingehen.
Davon speziell erklären, wie mit Binpacking eine minimale Anzahl an Rüstkosten erreicht werden kann für $s_{fg}=s$.


\section{Zulässigkeit der Zuweisung}
\textit{Bis Mitte Juli sollte die Lösung hier erarbeitet sein.}

Folgende Fragen diskutieren und idealer Weise beweisen:
Gibt es unzulässige Lösungen beim Binpacking? Wenn ja, wie lassen sie sich reparieren?


\section{Anordnung der Jobgruppen}
\subsection{MIP mit fixierten Bins}
Vorstellung des MIPs mit fixierten Bins mit $s_{fg}=s$. Vorstellung einiger Laufzeiten. 
\textit{MIP steht. Evtl müssen noch einige Laufzeiten gemessen werden.}

Außerdem die erweiterte Formulierung auf allgemeine $s_{fg}$ und Diskussion.
\textit{MIP sollte bis Ende Juli formuliert und Laufzeiten gemessen sein.}

\subsection{MIP mit freien Bins}
\label{subsec:mipfreibins}
\textit{Muss nur noch aufgeschrieben werden. Evtl. noch ein paar Laufzeiten messen.}

Vorstellung des MIPs mit freien Bins. 
Die Laufzeit ist sehr viel schlechter, die primale Lösung ist aber meist schon nach kurzer Zeit besser als beim MIP mit fixierten Bins.

\subsection{Mehrere fixierte Reihenfolgen}
\textit{Jetzt damit anfangen. Wahrscheinlich bis Mitte August.}

Basierend auf den Erkenntnissen aus \ref{subsec:mipfreibins} das MIP mit fixierten Bins für mehrere Binreihenfolgen laufen lassen.
Um nicht sämtliche Binreihenfolgen durchprobieren zu müssen,
ggf. mit bipartitem gewichteten Matching ein lokales Optimum zwischen zwei Bins suchen und daraus eine "`gute"' Binreihenfolge erstellen.


\chapter{Messergebnisse und Vergleiche}
\textit{Anfang September anfangen, soweit Messergebnisse vorliegen.}

Obwohl in den vorherigen Kapiteln schon einige Laufzeiten vorgestellt werden, hier nochmal eine Zusammenfassung und insbesondere ein Vergleich
zwischen den beiden Dekompositionsansätzen. Sowohl echte Instanzen als auch generierte. 
Dabei ggf. auf Methoden zur Instanzengenerierung eingehen und diese bewerten?

\end{document}
