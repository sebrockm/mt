\documentclass{article}

\usepackage[utf8]{inputenc}
\usepackage{amsmath}
\usepackage{amssymb}


\begin{document}

\section*{MIP zum Anordnen der Jobgruppen in den Bins}
Mengen\\
\begin{tabular}{lll}
    $N=\{1,\dots,n\}$ & & Jobgruppen\\
    $M=\{1,\dots,m\}$ & & Bins\\
    $N_b\subseteq N$ & $\forall b\in M$ & Jobgruppen in Bin $b$\\
    $N_b^*\subseteq N$ & $\forall b\in M$ & Jobgruppen im Nachbarbin $b\bmod m+1$
\end{tabular}%
\\
\\
Parameter\\
\begin{tabular}{lll}
    $B$ & & Größe der Bins\\
    $g_i$ & $i\in N$ & Größe von Jobgruppe $i$\\
    $p_{1i},p_{2i}$ & $i\in N$ & Prozesszeiten der Jobs in Gruppe $i$\\
    $b_i$ & $i\in N$ & Bin von Jobgruppe $i$
\end{tabular}%
\\
\\
Variablen\\
\begin{tabular}{lll}
    $x_{ij}\in\{0,1\}$ & $i\in N,j\in N_{b_i}, i\neq j$ &
            Jobgruppe $i$ liegt vor $j$\\
    $s_i\in\mathbb{R}$ & $i\in N$ & Startzeitpunkt der Jobgruppe $i$\\
    $o_{ij}\in\mathbb{R}$ & $i\in N,j\in N_{b_i}^*$ &
            Überschneidung von Jobgruppe $i$ und $j$ \\
    $\tilde{o}_{ij}\in\mathbb{R}$ & $i\in N,j\in N_{b_i}^*$ &
            Hilfsvariable für Überschneidung \\
    $y_{ij},z_{ij}\in\{0,1\}$ & $i\in N,j\in N_{b_i}^*$ & Indikatorvariablen
\end{tabular}
\\
\\
\\
%
%\begin{figure}[ht]
\begin{align}
    \text{min}\quad &\sum_{b\in M}\sum_{\substack{i\in N_b\\j\in N_b^*}} 
            \max\{p_{2i},p_{1j}\}o_{ij} \label{objfunc}\\
    \text{s.t.} \quad x_{ij} + x_{ji} &= 1 &i\in N, j\in N_{b_i} 
            \label{con:reflex}\\
    x_{ij} + x_{jk} &\leq x_{ik} + 1 &i\in N,\,\, j,k\in N_{b_i} 
            \label{con:transitiv} \\
    s_i &= \sum_{\substack{j\in N_{b_i}\\i\neq j}} g_j x_{ji} &i\in N 
            \label{con:start} \\
    My_{ij} &\geq s_j - s_i + \sigma_{ij}+ \epsilon &i\in N, j\in N^*_{b_i} \\
    M(1-y_{ij}) &\geq s_i - s_j - \sigma_{ij} &i\in N, j\in N^*_{b_i} \\
    Mz_{ij} &\geq s_i + g_i - s_j - g_j - \sigma_{ij} + \epsilon &i\in N, j\in N^*_{b_i} \\
    M(1-z_{ij}) &\geq s_j + g_j- s_i - g_i + \sigma_{ij} + \epsilon &i\in N, j\in N^*_{b_i} \\
    o_{ij} &\geq s_i + g_i - s_j - \sigma_{ij} - M(1-y_{ij}+z_{ij}) &i\in N, j\in N^*_{b_i} \\
    o_{ij} &\geq s_j + g_j - s_i + \sigma_{ij} - M(1-y_{ji}+z_{ji}) &i\in N, j\in N^*_{b_i} \\
    o_{ij} &\geq g_j - M(2-y_{ij}-z_{ij}) &i\in N, j\in N^*_{b_i} \\
    o_{ij} &\geq g_i - M(2-y_{ji}-z_{ji}) &i\in N, j\in N^*_{b_i} 
\end{align}
%\end{figure}%
%
Für je zwei Jobgruppen $i,j\in N_b$, die im selben Bin $b$ liegen, gibt es
zwei Variablen $x_{ij},x_{ji}$. Es ist $x_{ij}=1$ genau dann, wenn 
Jobgruppe $i$ vor $j$ auf dem Bin liegt. Dabei muss $i$ nicht notwendiger
Weise direkt vor $j$ liegen, sondern an beliebiger Position davor.
Mit (\ref{con:reflex}) wird erreicht, dass immer entweder $i$ vor $j$ liegt
oder umgekehrt. (\ref{con:transitiv}) stellt die Transitivität dieser Beziehung
her. Auf Grundlage der $x_{ij}$ und der Größen der Jobgruppen $g_i$ 
kann in (\ref{con:start}) der Startzeitpunkt jeder Jobgruppe berechnet werden.
Die Variablen $s_i$ dienen dabei nur der Übersicht und können in den übrigen 
Constraints durch die Summe aus (\ref{con:start}) ersetzt werden.

Über (\ref{con:y1}) und (\ref{con:y2}) wird den Indikatorvariablen $y_{ij}$ für
zwei Jobgruppen $i,j$ aus benachbarten Bins folgende
Bedeutung gegeben: $y_{ij}=1$ genau dann, wenn die Differenz $s_j+g_j-s_i$ zwischen dem
Endzeitpunkt von $j$ und dem Startzeitpunkt von $i$ größer ist als
die Differenz $s_i+g_i-s_j$ zwischen dem Endzeitpunkt von $i$ und dem Startzeitpunkt von $j$.
Dabei behandelt (\ref{con:y2}) den Spezialfall, dass $i$ in Bin $m$ liegt und
$j$ in Bin $1$, sodass jeweils eine Verschiebung der Start- und Endzeitpunkte
von $j$ um $1$ erforderlich ist.

Mit Hilfe der $y_{ij}$ bewirken die Constraints (\ref{con:maxij1}) bis
(\ref{con:maxji2}), dass die Variablen $\tilde{o}_{ij}$ jeweils das Minimum
der Differenzen aus Start- und Endzeitpunkten annehmen, also
$\tilde{o}_{ij}=\min\{s_j+g_j-s_i, s_i+g_i-s_j\}$.

Den Indikatorvariablen $z_{ij}$ wird für zwei Jobgruppen in benachbarten Bins
in (\ref{con:z}) folgende Bedeutung gegeben:
$z_{ij}=1 \iff \tilde{o}_{ij} > min\{g_i,g_j\}$. Die $z_{ij}$ geben also an, ob die
durch $\tilde{o}_{ij}$ gegebenen Überschneidungen größer sind als die Längen
der Jobgruppen. Mit (\ref{con:overlay1}) und (\ref{con:overlay2}) wird den
Variablen $o_{ij}$ dann die endgültige Länge der Überschneidung der Jobgruppen
$i$ und $j$ zugewiesen, nämlich 
$o_{ij} = \min\{\tilde{o}_{ij},g_i,g_j\} = \min\{s_j+g_j-s_i,s_i+g_i-s_j,g_i,g_j\}$.
Damit kann dann die Zielfunktion (\ref{objfunc}) formuliert werden als Summe
der Länge der Überschneidungen multipliziert mit den jeweiligen Prozesszeiten
der Jobs.
\end{document}
