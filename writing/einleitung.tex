\documentclass{article}

\usepackage[utf8]{inputenc}
\usepackage{amsmath}
\usepackage{amssymb}


\begin{document}

\section{Problembeschreibung}
Bei synchronen Flow-Shop-Problemen handelt es sich um Produktions\-planungs\-probleme,
bei denen die zu produzierenden Güter (Jobs) auf einer kreisförmigen rotierenden Produktionsanlage
produziert werden. Die Produktionsanlage be\-steht aus $m$ Stationen, die sich mit der Anlage drehen.
Außen, um die Anlage herum, befinden sich $m$ fortlaufend nummerierte fixierte Maschinen $M_1,\dots,M_m$, die die einzelnen Produktionsschritte durchführen.
Dabei handelt es sich bei Maschine $M_1$ um das Einlegen des Jobs in die Anlage und bei Maschine $M_m$ um die Entnahme des fertigen Produkts.
Durch Rotation der Anlage werden die Stationen mit den auf ihnen befindlichen Jobs zur jeweils nächsten Maschine transportiert.
Eine Rotation darf immer nur dann stattfinden, wenn alle Maschinen ihren Produktionsschritt an ihrem aktuellen Job
durchgeführt haben. Auf diese Weise können die Jobs nur \textit{synchron} zur nachfolgenden Maschine gelangen.
Die Zeit, die zwischen zwei Rotationen vergeht, wird als \textit{Zykluszeit} bezeichnet.

Die zu produzierenden Jobs sind gegeben durch die Menge $J=\{j_1,\dots,j_n\}$ 
und die Prozesszeiten von Job $j$ auf Maschine $i$ sind durch $p_{ij}$ gegeben.
Ziel ist es, eine Reihenfolge $\pi\in J^n$ der Jobs zu erstellen, der die gesamte Produktionsdauer minimiert.
Die Zykluszeiten $c_k$ mit $1\leq k\leq n+m-1$ berechnen sich wie folgt:
\[ c_k = \max_{i=\max\{1,k-n+1\}}^{\min\{k,m\}} p_{i\pi_{k-i+1}} \]
Zusätzlich benötigen die Jobs Ressourcen aus einer Menge $R$, um in die Stationen eingelegt werden zu können. Diese Ressourcen können erst nach
Fertigstellung eines Jobs, also nachdem er an Maschine $M_m$ aus der Anlage genommen wurde, wiederverwendet werden.
Sie sind allerdings nur in begrenzter Zahl vorhanden und im Allgemeinen ist nicht jede Ressource für jeden Job geeignet.
Für $j\in J$ sei $\rho_j\subseteq R$ die Menge der Ressourcen, die für $j$ geeignet ist.
Umgekehrt sei für $r\in R$ mit $\iota_r\subseteq J$ die Menge der Jobs bezeichnet, für die $r$ geeignet ist.
An Maschine $M_1$ kann es daher notwendig sein, vor dem Einlegen des nächsten Jobs die Ressource zu wechseln, 
wenn auf der entsprechenden Station zuvor Job $j\in J$ mit Ressource $r\in\rho_j$ fertiggestellt wurde 
und nun Job $j'\in J$ eingelegt werden soll, wobei $r\not\in\rho_{j'}$ ist.

Für die Ressourcen können folgende Situationen auftreten:
\begin{itemize}
    \item Alle Ressourcen sind für alle Jobs geeignet, also $\rho_j=R$ für alle $j\in J$.
    \item Die Jobs lassen sich in disjunkte Gruppen unterteilen, so dass für alle Jobs aus einer Gruppe dieselbe Ressourcenmenge geeignet ist.
        Wenn also $\rho_i \cap \rho_j \neq \emptyset$, dann folgt $\rho_i=\rho_j$.
    \item Die Ressourcenmengen bilden Hierarchien. 
        D.h., wenn $\iota_q \cap \iota_r \neq \emptyset$, dann folgt $\iota_q \subseteq \iota_r$ oder $\iota_r \subseteq \iota_q$.
    \item Die $\rho_j$ sind beliebige Teilmengen von $R$.
\end{itemize}%
Neben dem Wechsel von Ressourcen, der zusätzliche Zeit in Anspruch nimmt, können auch andere Formen von \textit{Rüstkosten}
auftreten. Z.B. kann es sein, dass an einer Station zunächst einige Umstellungen vorgenommen werden müssen, bevor der
neue Job eingelegt werden kann. Die Jobs können in Familien $\mathcal{F}$ eingeteilt werden, so dass beim Übergang
zwischen zwei Jobs aus den Familien $f$ und $g$ die Rüstkosten $s_{fg}$ auftreten.
Diese Rüstkosten können 
\begin{itemize}
    \item sowohl vom Vorgänger als auch vom Nachfolger abhängig sein ($s_{fg}$), 
    \item nur vom Nachfolger abhängig sein ($s_{fg} = s_{g}$) oder
    \item konstant sein ($s_{fg} = s > 0$).
\end{itemize}%
Dabei wird jeweils für $s_{ff} = 0$ angenommen, dass also keine Rüstkosten innerhalb einer Familie auftreten.

Insgesamt gilt es also, neben dem Schedule $\pi$ auch ein Mapping $f:J\rightarrow R$ zu finden, 
das jedem Job $j\in J$ eine Ressource $r\in\rho_j$ zuweist und folgenden Ansprüchen genügt:
\begin{itemize}
    \item $f$ muss zulässig sein in dem Sinne, dass beim Einlegen jedes Jobs $j\in J$ eine Ressource $r\in\rho_j$ verfügbar ist
        (d.h., dass $r$ sich nicht gerade an anderer Stelle in der Anlage befindet).
    \item $f$ und $\pi$ zusammen sollen optimal sein in dem Sinne, dass die Summe aus den durch $\pi$ und $f$ definierten Zykluszeiten und 
        Rüstkosten minimal ist.
\end{itemize}

\section{Motivation}
In \cite{...} wurde gezeigt, dass dieses Problem schon $\mathcal{NP}$-schwer ist, wenn alle Ressourcen für alle Jobs geeignet sind
und keine Rüstkosten auftreten.
Versuche, dieses Problem -- oder auch einige Spezialfälle davon -- mit linearer Programmierung zu lösen, waren nur für sehr kleine Instanzen mit
$n<30$ erfolgreich, was weit hinter praktischen Anforderungen zurückliegt. Aufgrund der Komlexität des Problems sollen in dieser Arbeit
zwei Dekompositionsansätze verfolgt werden:
\begin{enumerate}
    \item Zunächst wird eine Reihenfolge $\pi$ aufgestellt, ohne Ressourcen und Rüstkosten zu betrachten, 
        und anschließend wird das Mapping $f$ basierend auf $\pi$ erstellt, ohne $\pi$ noch zu verändern.
    \item Es wird zuerst das Mapping $f$ erstellt, so dass die Ressourcen zulässig zugewiesen sind und die Rüstkosten minimal sind.
        Anschließend werden, ohne $f$ zu verändern, die Jobs so angeordnet, dass die Zykluszeiten möglichst minimal sind, und so $\pi$ erstellt.
\end{enumerate}%
Beide Ansätze liefern natürlich im Allgemeinen keine optimalen Lösungen. 
Außerdem sind selbst die aus den Ansätzen resultierenden Teilprobleme teilweise noch $\mathcal{NP}$-schwer.
Beispielsweise ist bei Ansatz (1) das Berechnen einer optimalen Reihenfolge $\pi$ (soweit bekannt) nur in dem Spezialfall $m=2$ 
mit dem Algorithmus von Gilmore und Gomory \cite{...} in Polynomialzeit lösbar.
Bei der anschließenden Zuweisung von Ressourcen ist noch unbekannt, ob ein polynomieller Algorithmus existiert.
Dies herauszufinden ist eines der Ziele dieser Arbeit.

Da die Rüstkosten $s_{fg}$ in dem gegebenen Praxisfall deutlich größer sind als die Prozesszeiten $p_{ij}$ der Jobs auf den Maschinen,
ist davon auszugehen, dass Ansatz (2) für diesen Praxisfall die besseren Lösungen erzielen wird.
Nichtsdestotrotz ist auch der erste Ansatz interessant, da er in anderen Praxisbeispielen von Bedeutung sein kann.


\end{document}
